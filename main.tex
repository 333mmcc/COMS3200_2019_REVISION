%%%%%%%%%%%%%%%%%%%%%%%%%%%%%%%%%%%%%%%%%
% Lachaise Assignment
% LaTeX Template
% Version 1.0 (26/6/2018)
%
% This template originates from:
% http://www.LaTeXTemplates.com
%
% Authors:
% Marion Lachaise & François Févotte
% Vel (vel@LaTeXTemplates.com)
%
% License:
% CC BY-NC-SA 3.0 (http://creativecommons.org/licenses/by-nc-sa/3.0/)
% 
%%%%%%%%%%%%%%%%%%%%%%%%%%%%%%%%%%%%%%%%

%----------------------------------------------------------------------------------------
%	PACKAGES AND OTHER DOCUMENT CONFIGURATIONS
%----------------------------------------------------------------------------------------

\documentclass{article}

%%%%%%%%%%%%%%%%%%%%%%%%%%%%%%%%%%%%%%%%%
% Lachaise Assignment
% Structure Specification File
% Version 1.0 (26/6/2018)
%
% This template originates from:
% http://www.LaTeXTemplates.com
%
% Authors:
% Marion Lachaise & François Févotte
% Vel (vel@LaTeXTemplates.com)
%
% License:
% CC BY-NC-SA 3.0 (http://creativecommons.org/licenses/by-nc-sa/3.0/)
% 
%%%%%%%%%%%%%%%%%%%%%%%%%%%%%%%%%%%%%%%%%

%----------------------------------------------------------------------------------------
%	PACKAGES AND OTHER DOCUMENT CONFIGURATIONS
%----------------------------------------------------------------------------------------

\usepackage{amsmath,amsfonts,stmaryrd,amssymb,float,graphicx,enumerate,mathtools,listings,xcolor,color,esint,subfig,tikz,pgfplots,setspace,graphicx,array,longtable} % Math packages

\usepackage{enumerate} % Custom item numbers for enumerations

\usepackage[ruled]{algorithm2e} % Algorithms

\usepackage[framemethod=tikz]{mdframed} % Allows defining custom boxed/framed environments

\usepackage{listings} % File listings, with syntax highlighting
\lstset{
	basicstyle=\ttfamily, % Typeset listings in monospace font
}

%----------------------------------------------------------------------------------------
%	DOCUMENT MARGINS
%----------------------------------------------------------------------------------------

\usepackage{geometry} % Required for adjusting page dimensions and margins

\geometry{
	paper=a4paper, % Paper size, change to letterpaper for US letter size
	top=2.5cm, % Top margin
	bottom=3cm, % Bottom margin
	left=2.5cm, % Left margin
	right=2.5cm, % Right margin
	headheight=14pt, % Header height
	footskip=1.5cm, % Space from the bottom margin to the baseline of the footer
	headsep=1.2cm, % Space from the top margin to the baseline of the header
	%showframe, % Uncomment to show how the type block is set on the page
}

%----------------------------------------------------------------------------------------
%	FONTS
%----------------------------------------------------------------------------------------

\usepackage[utf8]{inputenc} % Required for inputting international characters
\usepackage[T1]{fontenc} % Output font encoding for international characters

\usepackage{XCharter} % Use the XCharter fonts

%----------------------------------------------------------------------------------------
%	COMMAND LINE ENVIRONMENT
%----------------------------------------------------------------------------------------

% Usage:
% \begin{commandline}
%	\begin{verbatim}
%		$ ls
%		
%		Applications	Desktop	...
%	\end{verbatim}
% \end{commandline}

\mdfdefinestyle{commandline}{
	leftmargin=10pt,
	rightmargin=10pt,
	innerleftmargin=15pt,
	middlelinecolor=black!50!white,
	middlelinewidth=2pt,
	frametitlerule=false,
	backgroundcolor=black!5!white,
	frametitle={Command Line},
	frametitlefont={\normalfont\sffamily\color{white}\hspace{-1em}},
	frametitlebackgroundcolor=black!50!white,
	nobreak,
}

% Define a custom environment for command-line snapshots
\newenvironment{commandline}{
	\medskip
	\begin{mdframed}[style=commandline]
}{
	\end{mdframed}
	\medskip
}

%----------------------------------------------------------------------------------------
%	FILE CONTENTS ENVIRONMENT
%----------------------------------------------------------------------------------------

% Usage:
% \begin{file}[optional filename, defaults to "File"]
%	File contents, for example, with a listings environment
% \end{file}

\mdfdefinestyle{file}{
	innertopmargin=1.6\baselineskip,
	innerbottommargin=0.8\baselineskip,
	topline=false, bottomline=false,
	leftline=false, rightline=false,
	leftmargin=2cm,
	rightmargin=2cm,
	singleextra={%
		\draw[fill=black!10!white](P)++(0,-1.2em)rectangle(P-|O);
		\node[anchor=north west]
		at(P-|O){\ttfamily\mdfilename};
		%
		\def\l{3em}
		\draw(O-|P)++(-\l,0)--++(\l,\l)--(P)--(P-|O)--(O)--cycle;
		\draw(O-|P)++(-\l,0)--++(0,\l)--++(\l,0);
	},
	nobreak,
}

% Define a custom environment for file contents
\newenvironment{file}[1][File]{ % Set the default filename to "File"
	\medskip
	\newcommand{\mdfilename}{#1}
	\begin{mdframed}[style=file]
}{
	\end{mdframed}
	\medskip
}

%----------------------------------------------------------------------------------------
%	NUMBERED QUESTIONS ENVIRONMENT
%----------------------------------------------------------------------------------------

% Usage:
% \begin{question}[optional title]
%	Question contents
% \end{question}

\mdfdefinestyle{question}{
	innertopmargin=1.2\baselineskip,
	innerbottommargin=0.8\baselineskip,
	roundcorner=5pt,
	nobreak,
	singleextra={%
		\draw(P-|O)node[xshift=1em,anchor=west,fill=white,draw,rounded corners=5pt]{%
		Question \theQuestion\questionTitle};
	},
}

\newcounter{Question} % Stores the current question number that gets iterated with each new question

% Define a custom environment for numbered questions
\newenvironment{question}[1][\unskip]{
	\bigskip
	\stepcounter{Question}
	\newcommand{\questionTitle}{~#1}
	\begin{mdframed}[style=question]
}{
	\end{mdframed}
	\medskip
}

%----------------------------------------------------------------------------------------
%	WARNING TEXT ENVIRONMENT
%----------------------------------------------------------------------------------------

% Usage:
% \begin{warn}[optional title, defaults to "Warning:"]
%	Contents
% \end{warn}

\mdfdefinestyle{warning}{
	topline=false, bottomline=false,
	leftline=false, rightline=false,
	nobreak,
	singleextra={%
		\draw(P-|O)++(-0.5em,0)node(tmp1){};
		\draw(P-|O)++(0.5em,0)node(tmp2){};
		\fill[black,rotate around={45:(P-|O)}](tmp1)rectangle(tmp2);
		\node at(P-|O){\color{white}\scriptsize\bf !};
		\draw[very thick](P-|O)++(0,-1em)--(O);%--(O-|P);
	}
}

% Define a custom environment for warning text
\newenvironment{warn}[1][Warning:]{ % Set the default warning to "Warning:"
	\medskip
	\begin{mdframed}[style=warning]
		\noindent{\textbf{#1}}
}{
	\end{mdframed}
}

%----------------------------------------------------------------------------------------
%	INFORMATION ENVIRONMENT
%----------------------------------------------------------------------------------------

% Usage:
% \begin{info}[optional title, defaults to "Info:"]
% 	contents
% 	\end{info}

\mdfdefinestyle{info}{%
	topline=false, bottomline=false,
	leftline=false, rightline=false,
	nobreak,
	singleextra={%
		\fill[black](P-|O)circle[radius=0.4em];
		\node at(P-|O){\color{white}\scriptsize\bf i};
		\draw[very thick](P-|O)++(0,-0.8em)--(O);%--(O-|P);
	}
}

% Define a custom environment for information
\newenvironment{info}[1][Info:]{ % Set the default title to "Info:"
	\medskip
	\begin{mdframed}[style=info]
		\noindent{\textbf{#1}}
}{
	\end{mdframed}
}

\pgfplotsset{compat=1.12}
\usepgfplotslibrary{fillbetween}
\DeclareMathOperator{\sech}{sech}
\DeclareMathOperator{\arcsech}{arcsech}
\DeclareMathOperator{\arcoth}{arcoth}
\DeclareMathOperator{\arctanh}{tanh^{-1}}
\renewcommand{\baselinestretch}{1.25}
\newcommand{\bin}{\sf Bin}
\newcommand{\G}{\sf G}
\newcommand{\Hyp}{\sf Hyp}
\newcommand{\Po}{\sf Po}
\newcommand{\ex}{\sf Exp}
\newcommand{\Nor}{\sf N}
\newcommand{\gam}{\sf Gam} 
\newcommand{\Em}{\mathbb E}
\newcommand{\Pm}{\mathbb P}
\newcommand{\R}{\mathbb R}
\newcommand{\B}{\cal B}
\newcommand{\N}{\mathbb N}
\newcommand{\Z}{\mathbb Z}
\newcommand{\Q}{\mathbb Q}
\newcommand{\Comp}{\mathbb C}
\newcommand{\e}{\mbox{e}}
\newcommand{\Var}{\mbox{Var}}
\newcommand{\cov}{\mbox{Cov}}
\newcommand{\ds}{\displaystyle}
\newcommand{\subvect}[1]{\mbox{\small \boldmath #1}}
%\newcommand{\vect}[1]{\mbox{\boldmath $#1$}}
\newcommand{\vect}[1]{\boldsymbol #1}
\newcommand{\norm}[1]{\left\lVert#1\right\rVert}
\newcommand*\VF[1]{\mathbf{#1}}
\newcommand*\dif{\mathop{}\!\mathrm{d}}
\setlength{\jot}{7pt}
\definecolor{lbcolor}{rgb}{0.95,0.95,0.95}

\title{COMS3200: Revision} % Title of the assignment
\author{The bois}
\date{University of Queensland --- \today}
\begin{document}

\maketitle % Print the title

\section*{Ch 1. Introduction}
\noindent
\rule{\linewidth}{0.5mm}
\noindent

\subsection*{Misc definitions, Sections 1.1 - 1.2.1}
\begin{description}
    \item[Internet] The Internet is the global system of interconnected computer networks that use the Internet protocol suite (TCP/IP) to link devices worldwide. It is a network of networks that consists of private, public, academic, business, and government networks of local to global scope, linked by a broad array of electronic, wireless, and optical networking technologies. The Internet carries a vast range of information resources and services, such as the inter-linked hypertext documents and applications of the World Wide Web (WWW), electronic mail, telephony, and file sharing.
    
    \item[Network Hosts/End systems] A network host is a computer connected to a computer network. A host may work as a server offering information resources, services, and applications to users or other nodes on the network. Hosts are assigned at least one network address.
    
    \item[Internet Protocol (IP)] The Internet Protocol (IP) is the principal communications protocol in the Internet protocol suite for relaying datagrams across network boundaries. Its routing function enables internetworking, and essentially establishes the Internet. IP has the task of delivering packets from the source host to the destination host solely based on the IP addresses in the packet headers. For this purpose, IP defines packet structures that encapsulate the data to be delivered. It also defines addressing methods that are used to label the datagram with source and destination information. Historically, IP was the connectionless datagram service in the original Transmission Control Program introduced by Vint Cerf and Bob Kahn in 1974, which was complemented by a connection-oriented service that became the basis for the Transmission Control Protocol (TCP). The Internet protocol suite is therefore often referred to as TCP/IP. The first major version of IP, Internet Protocol Version 4 (IPv4), is the dominant protocol of the Internet. Its successor, Internet Protocol Version 6 (IPv6), has been growing in adoption, 

    \item[Distributed Computing] Distributed computing is a field of computer science that studies distributed systems. A distributed system is a system whose components are located on different networked computers, which communicate and coordinate their actions by passing messages to one another. The components interact with one another in order to achieve a common goal. Three significant characteristics of distributed systems are: concurrency of components, lack of a global clock, and independent failure of components. Examples of distributed systems vary from SOA-based systems to massively multiplayer online games to peer-to-peer applications.
     
    \item[Network Protocols] A network protocol defines the format and the order of messages exchanged between
    two or more communicating entities, as well as the actions taken on the transmission
    and/or receipt of a message or other event.

    \item[Clients] A client is a computer or a program that, as part of its operation, relies on sending a request to another program or a computer hardware or software that accesses a service made available by a server(which may or may not be located on another computer). For example, web browsers are clients that connect to web servers and retrieve web pages for display. Email clients retrieve email from mail servers. Online chat uses a variety of clients, which vary depending on the chat protocol being used. Multiplayer video games or online video games may run as a client on each computer. The term "client" may also be applied to computers or devices that run the client software or users that use the client software.
    
    \item[Servers] In computing, a server is a computer program or a device that provides functionality for other programs or devices, called "clients". This architecture is called the client–server model, and a single overall computation is distributed across multiple processes or devices. Servers can provide various functionalities, often called "services", such as sharing data or resources among multiple clients, or performing computation for a client. A single server can serve multiple clients, and a single client can use multiple servers. A client process may run on the same device or may connect over a network to a server on a different device. Typical servers are database servers, file servers, mail servers, print servers, web servers, game servers, and application servers.
    
    \item[Client-Server model] Client–server model is a distributed application structure that partitions tasks or workloads between the providers of a resource or service, called servers, and service requesters, called clients. Often clients and servers communicate over a computer network on separate hardware, but both client and server may reside in the same system. A server host runs one or more server programs which share their resources with clients. A client does not share any of its resources, but requests a server's content or service function. Clients therefore initiate communication sessions with servers which await incoming requests. Examples of computer applications that use the client–server model are Email, network printing, and the World Wide Web.
    
    \item[Digital Subscriber Line (DSL)] Digital subscriber line (DSL; originally digital subscriber loop) is a family of technologies that are used to transmit digital data over telephone lines. In telecommunications marketing, the term DSL is widely understood to mean asymmetric digital subscriber line (ADSL), the most commonly installed DSL technology, for Internet access. DSL service can be delivered simultaneously with wired telephone service on the same telephone line since DSL uses higher frequency bands for data. On the customer premises, a DSL filter on each non-DSL outlet blocks any high-frequency interference to enable simultaneous use of the voice and DSL services. A residence typically obtains DSL Internet access
    from the same local telephone company (telco) that provides its wired local phone
    access. Thus, when DSL is used, a customer’s telco is also its ISP. Eeach customer’s DSL modem uses the existing telephone line (twistedpair copper wire, which we’ll discuss in Section 1.2.2) to exchange data with a digital
    subscriber line access multiplexer (DSLAM) located in the telco’s local central
    office (CO). The home’s DSL modem takes digital data and translates it to highfrequency
    tones for transmission over telephone wires to the CO; the analog signals
    from many such houses are translated back into digital format at the DSLAM.
    The residential telephone line carries both data and traditional telephone signals
    simultaneously, which are encoded at different frequencies:
    \begin{itemize}
        \item A high-speed downstream channel, in the 50 kHz to 1 MHz band
        \item A medium-speed upstream channel, in the 4 kHz to 50 kHz band
        \item An ordinary two-way telephone channel, in the 0 to 4 kHz band
    \end{itemize}
    This approach makes the single DSL link appear as if there were three separate
    links, so that a telephone call and an Internet connection can share the DSL link at
    the same time. On the customer side, a splitter separates the data and telephone signals
    arriving to the home and forwards the data signal to the DSL modem. On the
    telco side, in the CO, the DSLAM separates the data and phone signals and sends
    the data into the Internet. Hundreds or even thousands of households connect to a
    single DSLAM [Dischinger 2007]. The DSL standards define transmission rates of 12 Mbps downstream and
    1.8 Mbps upstream [ITU 1999], and 24 Mbps downstream and 2.5 Mbps upstream
    [ITU 2003]. Because the downstream and upstream rates are different, the access is
    said to be asymmetric. The actual downstream and upstream transmission rates
    achieved may be less than the rates noted above, as the DSL provider may purposefully
    limit a residential rate when tiered service (different rates, available at different
    prices) are offered, or because the maximum rate can be limited by the distance
    between the home and the CO, the gauge of the twisted-pair line and the degree of
    electrical interference. Engineers have expressly designed DSL for short distances
    between the home and the CO; generally, if the residence is not located within 5 to 10
    miles of the CO, the residence must resort to an alternative form of Internet access.

    \item[Cable Internet access] In telecommunications, cable Internet access, shortened to cable Internet, is a form of broadband Internet access which uses the same infrastructure as a cable television. Like digital subscriber line and fiber to the premises services, cable Internet access provides network edge connectivity (last mile access) from the Internet service provider to an end user. It is integrated into the cable television infrastructure analogously to DSL which uses the existing telephone network. Cable TV networks and telecommunications networks are the two predominant forms of residential Internet access. Cable Internet access makes use of the cable television company’s existing cable
    television infrastructure. A residence obtains cable Internet access from the same
    company that provides its cable television. Fiber optics
    connect the cable head end to neighborhood-level junctions, from which traditional
    coaxial cable is then used to reach individual houses and apartments. Each
    neighborhood junction typically supports 500 to 5,000 homes. Because both fiber
    and coaxial cable are employed in this system, it is often referred to as hybrid
    fiber coax (HFC). One important characteristic of cable Internet access is that it is a shared
    broadcast medium. In particular, every packet sent by the head end travels downstream
    on every link to every home and every packet sent by a home travels on the
    upstream channel to the head end. For this reason, if several users are simultaneously
    downloading a video file on the downstream channel, the actual rate at which
    each user receives its video file will be significantly lower than the aggregate cable
    downstream rate. On the other hand, if there are only a few active users and they
    are all Web surfing, then each of the users may actually receive Web pages at the
    full cable downstream rate, because the users will rarely request a Web page at
    exactly the same time. Because the upstream channel is also shared, a distributed
    multiple access protocol is needed to coordinate transmissions and avoid collisions.
\end{description}

\subsection*{Physical Media (Section 1.2.2)}

\begin{description}
    \item[Twisted-Pair Copper Wire] The least expensive and most commonly used guided transmission medium is
    twisted-pair copper wire. For over a hundred years it has been used by telephone
    networks. In fact, more than 99 percent of the wired connections from the telephone
    handset to the local telephone switch use twisted-pair copper wire. Most of
    us have seen twisted pair in our homes and work environments. Twisted pair consists
    of two insulated copper wires, each about 1 mm thick, arranged in a regular
    spiral pattern. The wires are twisted together to reduce the electrical interference
    from similar pairs close by. Typically, a number of pairs are bundled together in a
    cable by wrapping the pairs in a protective shield. A wire pair constitutes a single
    communication link. {\bf Unshielded twisted pair (UTP)} is commonly used for computer networks within a building, that is, for LANs. Data rates for LANs
    using twisted pair today range from 10 Mbps to 10 Gbps. The data rates that can
    be achieved depend on the thickness of the wire and the distance between transmitter
    and receiver.

    \item[Coaxial Cable] Like twisted pair, coaxial cable consists of two copper conductors, but the two conductors are concentric rather than parallel. With this construction and special insulation
    and shielding, coaxial cable can achieve high data transmission rates. Coaxial
    cable is quite common in cable television systems. As we saw earlier, cable television
    systems have recently been coupled with cable modems to provide residential
    users with Internet access at rates of tens of Mbps. In cable television and cable
    Internet access, the transmitter shifts the digital signal to a specific frequency band,
    and the resulting analog signal is sent from the transmitter to one or more receivers.
    Coaxial cable can be used as a guided shared medium. Specifically, a number of
    end systems can be connected directly to the cable, with each of the end systems
    receiving whatever is sent by the other end systems.

    \item[Fiber Optics] An optical fiber is a thin, flexible medium that conducts pulses of light, with each
    pulse representing a bit. A single optical fiber can support tremendous bit rates, up
    to tens or even hundreds of gigabits per second. They are immune to electromagnetic
    interference, have very low signal attenuation up to 100 kilometers, and are
    very hard to tap. These characteristics have made fiber optics the preferred longhaul
    guided transmission media, particularly for overseas links. Many of the longdistance
    telephone networks in the United States and elsewhere now use fiber optics
    exclusively. Fiber optics is also prevalent in the backbone of the Internet. However,
    the high cost of optical devices—such as transmitters, receivers, and switches—has
    hindered their deployment for short-haul transport, such as in a LAN or into the
    20 CHAPTER 1 • COMPUTER NETWORKS AND THE INTERNET
    home in a residential access network. The Optical Carrier (OC) standard link speeds
    range from 51.8 Mbps to 39.8 Gbps; these specifications are often referred to as OCn,
    where the link speed equals n × 51.8 Mbps.

    \item[Terrestrial Radio Channels] Radio channels carry signals in the electromagnetic spectrum. They are an attractive
    medium because they require no physical wire to be installed, can penetrate walls,
    provide connectivity to a mobile user, and can potentially carry a signal for long distances.
    The characteristics of a radio channel depend significantly on the propagation
    environment and the distance over which a signal is to be carried. Environmental considerations
    determine path loss and shadow fading (which decrease the signal strength
    as the signal travels over a distance and around/through obstructing objects), multipath
    fading (due to signal reflection off of interfering objects), and interference (due
    to other transmissions and electromagnetic signals).
    Terrestrial radio channels can be broadly classified into three groups: those that
    operate over very short distance (e.g., with one or two meters); those that operate in
    local areas, typically spanning from ten to a few hundred meters; and those that
    operate in the wide area, spanning tens of kilometers. Personal devices such as wireless
    headsets, keyboards, and medical devices operate over short distances; the
    wireless LAN technologies use local-area radio channels;
    the cellular access technologies use wide-area radio channels.

    \item[Satellite Radio Channels] A communication satellite links two or more Earth-based microwave transmitter/
    receivers, known as ground stations. The satellite receives transmissions on one frequency
    band, regenerates the signal using a repeater (discussed below), and transmits
    the signal on another frequency. Two types of satellites are used in communications:
    geostationary satellites and low-earth orbiting (LEO) satellites.
    Geostationary satellites permanently remain above the same spot on Earth. This
    stationary presence is achieved by placing the satellite in orbit at 36,000 kilometers
    above Earth’s surface. This huge distance from ground station through satellite back
    to ground station introduces a substantial signal propagation delay of 280 milliseconds.
    Nevertheless, satellite links, which can operate at speeds of hundreds of Mbps,
    are often used in areas without access to DSL or cable-based Internet access.
    LEO satellites are placed much closer to Earth and do not remain permanently
    above one spot on Earth. They rotate around Earth (just as the Moon does) and may
    communicate with each other, as well as with ground stations.
\end{description}
\end{document}
